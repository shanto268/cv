\cvsection{Employment}
\begin{cventries}
    \cventry
    {Advanced Particle Detector Laboratory (APDL)}
    {Undergraduate Research Assistant}
    {Lubbock, TX, USA}
    {Nov. 2018 - Present}
    {
        \begin{cvitems}
        \item{Objective: Develop portable muon telescope capable of 0.5 milliradian resolution imaging capability} 
        \item{Aided in the mechanical assembly of the phase 1 muon telescope}
        \item{Designed custom Winston Cone light collectors for increased optical transmission}
        \item{Facilitated the design of custom PCB's and assembled various components}
        \item{Implemented a multi-thread sync mechanism in the DAQ system comprised of 40 Arduino's and CAMAC systems}
        \item{Engineered the calibration and installation of 40 SiPM's (Phase 1) and 44 PMTs (Phase 2) on the telescope}
        % need to combine 
        %\item{Created Monte Carlo simulations to test experimental data integrity and measure phase 1 telescope efficiency}
        %\item{Upgraded the simulation software to include custom test cases, theorized designs and phase 2 telescope simulation}
        %\item{Programmed an automated data analysis program to extract key information from experimental data}    
        % need to combine 
        \item{Designed and implemented Monte Carlo simulation and automated analysis program to test experimental data integrity, assess theorized designs and meausre telescope efficiency}
        \item{Conducted Monte Carlo studies on the scattering/absorption behaviour of muons and the consequent effects in image quality}
        \item{Deployed all software used by the lab on our university's High Performance Computing Cluster}
        \item{Trained new undergraduate members in the lab to use Geant4, ROOT, and our custom software base}
        \item{Assisted with the tomogram generation algorithm using the muon tracks}
        \item{Coauthored the proposal for IRIS-HEP Fellows Program}
        \item{Currently incorporating concepts of image segmentation and ML to enhance final image and improve muon track reconstruction efficiency}
        \item{Supervisors: Shuichi Kunori, PhD. \& Nural Akchurin, PhD.}
        \end{cvitems}
        \vspace{1em}
    }

    \cventry
    {Texas Tech University}
    {Teaching Assistant, \textit{"Introduction to Quantum Information and Computation (QIC)"}}
    {Lubbock, TX, USA}
    {Aug. 2020 - Present}
    {
        \begin{cvitems}
        \item{Delivered supplemental lecture notes and interactive jupyter notebooks to teach quantum computing through the use of IBM's qiskit}
        \item{Prepared bi-weekly computational assignments on the implementation of various Quantum Information and Computing topics}
        \item{Helped students with their problems during office hours each week}
        \item{Graded both computational and theoretical/mathematical assignments for the 25+ students enrolled in the course}
        \item{Assisted and collaborated with the students in their semester research project}
        \item{Materials covered: \textit{qiskit API, single and multi qubit systems, statevector evolution, superposition and entanglement, quantum circuit model, quantum teleportation, Deutsch's algorithm, Deutsch-Jozsa Algorithm, Grover's Algorithm, Bernstein-Vazirani algorithm, VQE, and Jordan's Algorithm}}
        \item{Supervising Professor: Ismael Regis de-Farias, PhD.}
        \end{cvitems}
        \vspace{1em}
    }

    %\cventry
    %{Institute for Software Integrated Systems (ISIS), Vanderbilt University}
    %{Summer Research Intern}
    %{Nashville, TN, USA}
    %{Jun. -- Aug. 2020}
    %{
        %\begin{cvitems}
        %\item{Designed computationally efficient models for various microscopic traffic simulations using a system written in C\texttt{++}, Python, Bash and XML.}
        %\item{Contributed to developing a computational framework (Flow by UC Berkely) for deep RL and control experiments for traffic microsimulation.}
        %\item{Established an objected oriented system for calibrating results from stochastic simulations under multi-objective methods using gradient free algorithms.}
        %\item{Incorporated Ray to the software package to parallelize the simulations resulting in massive speedup of running simulation experiments}
        %\item{Developed scripts to convert microscopic data from the Intelligent Driver Model (IDM) to RDS/radar style data.}
        %\item{Implemented various non-trivial optimization routines to fit simulation data to macroscopic RDS data sets.}
        %\item{Studied the various challenges of Microsimulation Calibration with Traffic Waves using Aggregate Measurements and co-authored a conference paper.}
        %\item{Supervisors: Daniel Work, PhD. \& George Gunter (PhD Candidate) }
        %\end{cvitems}
    %}



    \cventry
    {Texas Tech Multidisciplinary Research in Transportation (TechMRT)}
    {Undergraduate Research Assistant}
    {Lubbock, TX, USA}
    {Jan. 2019 - Jun. 2020}
    {
        \begin{cvitems}
        \item{Project 1: Develop a customisable analysis and simulation software for studying various heterogeneous traffic flow of Human Driven (HVs) and Autonomous Vehicles (AVs)}
        \item{Project 2: Design and test various AV models for efficient shared lane mobility in multi-lane networks using a novel approach based on the Nagel-Schreckenberg Cellular Automaton Model}
        \item{Project 3: Incorporate Reinforcement Learning functionality to the simulation and analysis software (incomplete - Covid 19)}
    %\item{Interpreted and analyzed the results of various experiments led by Dr. Li and reported work in poster presentations}
    %\item{A comprehensive journal paper is in works presently}
        \item{Supervisor: Jia Li, PhD.}
        \end{cvitems}
        \vspace{1em}
    }

    \cventry
    {TECHniques Center}
    {STEM Peer Tutor}
    {Lubbock, TX, USA}
    {Jan. 2018 - May 2019}
    {
        \begin{cvitems}
        \item{Provided course-specific tutoring to undergraduate students with documented evidence of learning disabilities}
        \item{Received Level 2 International Tutor Certification from College Reading \& Learning Association (CRLA)}
        \item{Documented over 670 hours of student tutoring while maintaining federal confidentiality guidelines}
        \item{Courses tutored: \textit{Physics I and II, Calculus I and II, Circuits I, Object Oriented Programming, Wind Energy, Linear Algebra, Advanced Calculus, Differential Equations, Combinatorics and Statistics}}
        \end{cvitems}
        \vspace{1em}
    }

    \cventry
    {TexPREP (Prefreshman Engineering Program) Lubbock}
    {Course Instructor}
    {Lubbock, TX, USA}
    {May 2019 - Jul. 2019}
    {
        \begin{cvitems}
        \item{Taught advanced programming principles - data types, variables, control flow theory, compilers, loops, animation, game design, booleans, discrete numerical analysis - to middle school students on MIT’s Scratch IDE.}
        \item{Administered the after-school tutoring program by leading and training a group of Assistants.}
        \end{cvitems}
        \vspace{1em}
    }

\end{cventries}
