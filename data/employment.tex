\cvsection{Employment}
\begin{cventries}
    \cventry
    {Advanced Particle Detector Laboratory (APDL)}
    {Undergraduate Research Assistant}
    {Lubbock, TX, USA}
    {Nov. 2018 - Present}
    {
        \begin{cvitems}
        \item{Objective: Develop prototype for portable muon telescope capable of 0.5 milliradian resolution imaging capability} 
        \item{Aided in the mechanical assembly of the phase 1 muon telescope}
        \item{Designed custom Winston Cone light collectors for increased optical transmission in the telescope}
        \item{Facilitated the design of custom PCB's and helped soldered various components on them}
        \item{Implemented a multi-thread sync mechanism in the Data Acquisition System comprised of 40 Arduino's}
        \item{Engineered the calibration and installation of 40 SiPM's (Phase 1) and 20 PMT's (Phase 2) on the telescope}
        \item{Created Monte Carlo simulations to test experimental data integrity and measure phase 1 telescope efficiency}
        \item{Upgraded the simulation software to include custom test cases, theorized designs and phase 2 telescope simulation}
        \item{Programmed an automated data analysis program to extract key information from experimental data}    
        \item{Conducted Monte Carlo studies on the scattering/absorption behaviour of muons and the consequent effects in image quality}
        \item{Deployed all software used by the lab on our university's computing cluster}
        \item{Trained new undergraduate members in the lab to use Geant4, ROOT, and our custom software base}
        \item{Assisted with the tomograph generation algorithm using the muon trajectory data}
        \item{Currently, trying to incorporate concepts of image segmentation and ML to enhance final image and improve muon track reconstruction efficiency}
        \item{Presently, also working on installing various CRATES for the CAMAC system used for Phase 2 prototype}
        \item{Supervisors: Shuichi Kunori, PhD. \& Nural Akchurin, PhD.}
        \end{cvitems}
        \vspace{1em}
    }

    \cventry
    {Department of  Industrial, Manufacturing, and Systems Engineering (IMSE), Texas Tech University}
    {Teaching Assistant, \textit{"Introduction to Quantum Information and Computation (QIC)"}}
    {Lubbock, TX, USA}
    {Aug. 2020 - Present}
    {
        \begin{cvitems}
        \item{Create and deliver biweekly supplementary lectures for this graduate level introductory quantum computing course}
        \item{Prepare weekly computational assignments for the students to improve their knowledge of QIC, qiskit and python}
        \item{Help students with their problems during office hours each week}
        \item{Grade both computational and theoretical/mathematical assignments for the 25+ students enrolled in the course}
        \item{Proffessor: Ismael Regis de-Farias, PhD.}
        \end{cvitems}
        \vspace{1em}
    }

    \cventry
    {Texas Tech Multidisciplinary Research in Transportation (TechMRT)}
    {Undergraduate Research Assistant}
    {Lubbock, TX, USA}
    {Jan. 2019 - Jun. 2020}
    {
        \begin{cvitems}
        \item{Project 1: Develop a customisable analysis and simulation software for studying various heterogenous traffic flow of Human Driven (HVs) and Autonomous Vehicles (AVs)}
        \item{Project 2: Design and Test out various AV models for efficient shared lane mobility in multi-lane networks using a novel approach based on the Nagel-Schreckenberg Cellular Automaton Model}
        \item{Project 3: Incorporte Reinforcement Learning functionality to the simulation and analysis software (incomplete - Covid19)}
    \item{Interpreted and analyzed the results of various experiments led by Dr. Li and reported work in poster presentations}
    \item{A comprehensive journal paper is in works presently}
        \item{Supervisor: Jia Li, PhD.}
        \end{cvitems}
        \vspace{1em}
    }

    \cventry
    {TECHniques Center}
    {STEM Peer Tutor}
    {Lubbock, TX, USA}
    {Jan. 2018 - Jun. 2019}
    {
        \begin{cvitems}
        \item{Provided course-specific tutoring to undergraduate students with documented evidence of learning disabilities}
        \item{Received Level 2 International Tutor Certification from College Reading \& Learning Association (CRLA)}
        \item{Documented over 670 hours of student tutoring while maintaining federal confidentiality guidelines}
        \item{Courses tutored: \textit{Physics I and II, Calculus I and II, Circuits I, Object Oriented Programming, Wind Energy, Linear Algebra, Advanced Calculus, Differential Equations, Combinatorics and Statistics}}
        \end{cvitems}
        \vspace{1em}
    }

    \cventry
    {TexPREP (Prefreshman Engineering Program) Lubbock}
    {Course Instructor}
    {Lubbock, TX, USA}
    {May 2019 - Jul. 2019}
    {
        \begin{cvitems}
        \item{Taught advanced programming principles - data types, variables, control flow theory, compilers, loops, animation, game design, booleans, discrete numerical analysis - to middle school students on MIT’s Scratch IDE.}
        \item{Administered the after-school tutoring program by leading and training a group of Assistants.}
        \end{cvitems}
        \vspace{1em}
    }

\end{cventries}
