\cvsection{Employment}

\begin{cventries}

    \cventry
    {Levenson-Falk Lab (LFL)}
    {Graduate Research Assistant}
    {Los Angeles, CA, USA}
    {Jan. 2022 - Present}
    {
        \begin{cvitems}
        \item{Advisor: Eli Levenson-Falk PhD.}
        \end{cvitems}
        \vspace{1em}
    }


    \cventry
    {University of Southern California}
    {Teaching Assistant}
    {Los Angeles, CA, USA}
    {Aug. 2021 - May 2022}
    {
        \begin{cvitems}
        \item{Mentored and led 36 undergraduate engineering students for the lab section for the \textit{"Fundamentals of Physics II: Electricity and Magnetism"} course}
        %\item{Materials covered: \textit{physical circuit implementation; experimental verification of the existence of EM fields, potentials, Gauss' law and Faraday's Law; use of oscilloscope and function generators; analysis of RC \& LC circuits, transformers and toroids; study of resonance; calculation of Planck's constant}}
        %\item{Supervising Professor: \textit{G\"{o}khan Esirgen}, PhD.}
        \end{cvitems}
        \vspace{1em}
    }


    \cventry
    {Advanced Particle Detector Laboratory (APDL)}
    {Undergraduate Research Assistant}
    {Lubbock, TX, USA}
    {Nov. 2018 - Aug. 2021}
    {
        \begin{cvitems}
        \item{Led a team of 3 Summer Interns to use Machine Learning to develop auto-focus, depth perception and non-linear Filtered Back Propagation algorithms in the field of Muon Tomography}
        \item{Developed a Neural Network Architecture (Asymmetric Deep Mixture Density NN) that predicts muon hit locations from photon time propagation with a 87\% accuracy}
        \item{Designed a 3D reconstruction algorithm that uses CNN's to approximate a binary focus metric and dynamic k-means clustering with Image Segmentation and homomorphic transforms}
        \item{Designed and implemented Monte Carlo simulations (Geant4, ROOT) and wrote fully automated analysis programs (python) to test experimental data integrity, assess theorized designs and measure telescope efficiency}
        %\item{Deployed a web based 3D interactive Event Display system for our muon telescope system (WebGL, JS)}
        \item{Conducted Monte Carlo studies on the scattering/absorption behaviour of muons and the consequent effects in image quality}
        \item{Refactored and deployed all software used by the lab on our university's High Performance Computing (HPC) Cluster}
        \item{Engineered the calibration and installation of 40 SiPM's (Phase 1) and 44 PMTs (Phase 2) on the telescopes}
        %\item{Implemented a multi-thread sync mechanism (python and Arduino) in the DAQ system comprised of 40 Arduino's and CAMAC systems}
        \item{Facilitated the design of custom PCB's (kiCAD, LTspice) and designed (CAD and CNC machines) custom Winston Cone light collectors for increased optical transmission from Scintillators to SiPM array} 
        %\item{Designed (CAD and CNC machines) custom Winston Cone light collectors for increased optical transmission from Scintillators to SiPM array}
        %\item{Aided (welding and CAD designs) in the mechanical assembly of two prototype muon telescopes}
        %\item{Trained new undergraduate members in the lab to use Geant4, ROOT, and our custom software base}
        % need to combine 
        %\item{Created Monte Carlo simulations to test experimental data integrity and measure phase 1 telescope efficiency}
        %\item{Upgraded the simulation software to include custom test cases, theorized designs and phase 2 telescope simulation}
        %\item{Programmed an automated data analysis program to extract key information from experimental data}    
        % need to combine 
        %\item{Currently incorporating concepts of image segmentation and ML to enhance final image and improve muon track reconstruction efficiency}
        %\item{Objective: Develop portable muon telescope capable of 0.5 milliradian resolution imaging capability} 
        %\item{Coauthored the proposal for IRIS-HEP Fellows Program}
        \item{Supervisors: Shuichi Kunori, PhD. \& Nural Akchurin, PhD.}
        \end{cvitems}
        \vspace{1em}
    }

    \cventry
    {Texas Tech University}
    {Teaching Assistant, \textit{"Introduction to Quantum Information and Computation (QIC)"}}
    {Lubbock, TX, USA}
    {Aug. 2020 - Dec. 2020}
    {
        \begin{cvitems}
        \item{Delivered supplemental lecture notes and interactive jupyter notebooks to teach quantum computing through the use of IBM's qiskit}
        \item{Prepared bi-weekly computational assignments on the implementation of various Quantum Information and Computing topics}
        \item{Helped students with their problems during office hours each week}
        \item{Graded both computational and theoretical/mathematical assignments for the 25+ students enrolled in the course}
        \item{Assisted and collaborated with the students in their semester research project}
        \item{Materials covered: \textit{qiskit API, single and multi qubit systems, statevector evolution, superposition and entanglement, quantum circuit model, quantum teleportation, Deutsch's algorithm, Deutsch-Jozsa Algorithm, Grover's Algorithm, Bernstein-Vazirani algorithm, VQE, and Jordan's Algorithm}}
        \item{Supervising Professor: Ismael Regis de-Farias, PhD.}
        \end{cvitems}
        \vspace{1em}
    }

    %\cventry
    %{Institute for Software Integrated Systems (ISIS), Vanderbilt University}
    %{Summer Research Intern}
    %{Nashville, TN, USA}
    %{Jun. -- Aug. 2020}
    %{
        %\begin{cvitems}
        %\item{Designed computationally efficient models for various microscopic traffic simulations using a system written in C\texttt{++}, Python, Bash and XML.}
        %\item{Contributed to developing a computational framework (Flow by UC Berkely) for deep RL and control experiments for traffic microsimulation.}
        %\item{Established an objected oriented system for calibrating results from stochastic simulations under multi-objective methods using gradient free algorithms.}
        %\item{Incorporated Ray to the software package to parallelize the simulations resulting in massive speedup of running simulation experiments}
        %\item{Developed scripts to convert microscopic data from the Intelligent Driver Model (IDM) to RDS/radar style data.}
        %\item{Implemented various non-trivial optimization routines to fit simulation data to macroscopic RDS data sets.}
        %\item{Studied the various challenges of Microsimulation Calibration with Traffic Waves using Aggregate Measurements and co-authored a conference paper.}
        %\item{Supervisors: Daniel Work, PhD. \& George Gunter (PhD Candidate) }
        %\end{cvitems}
    %}



    \cventry
    {Texas Tech Multidisciplinary Research in Transportation (TechMRT)}
    {Undergraduate Research Assistant}
    {Lubbock, TX, USA}
    {Jan. 2019 - Jun. 2020}
    {
        \begin{cvitems}
        \item{Developed an \href{https://github.com/shanto268/comprehensive_simulation_traffic_analysis_software}{open source analysis and simulation software} for studying various heterogeneous traffic flow of Human Driven (HVs) and Autonomous Vehicles (AVs)}
        \item{Designed and tested various AV models for efficient shared lane mobility in multi-lane networks using a novel approach based on the Nagel-Schreckenberg Cellular Automaton Model}
        \item{Observed and explained intelligent herding phenomena in certain regimes of heterogeneous traffic flow in a journal paper}
        \item{Incorporated Reinforcement Learning functionality to the simulation and analysis software}
        \item{Supervisor: Jia Li, PhD.}
        \end{cvitems}
        \vspace{1em}
    }

    \cventry
    {TECHniques Center}
    {STEM Peer Tutor}
    {Lubbock, TX, USA}
    {Jan. 2018 - May 2019}
    {
        \begin{cvitems}
        \item{Provided course-specific tutoring to undergraduate students with documented evidence of learning disabilities}
        \item{Received Level 2 International Tutor Certification from College Reading \& Learning Association (CRLA)}
        \item{Documented over 670 hours of student tutoring while maintaining federal confidentiality guidelines}
        \item{Courses tutored: \textit{Physics I and II, Calculus I and II, Circuits I, Object Oriented Programming, Wind Energy, Linear Algebra, Advanced Calculus, Differential Equations, Combinatorics and Statistics}}
        \end{cvitems}
        \vspace{1em}
    }

    \cventry
    {TexPREP (Prefreshman Engineering Program) Lubbock}
    {Course Instructor}
    {Lubbock, TX, USA}
    {May 2019 - Jul. 2019}
    {
        \begin{cvitems}
        \item{Taught advanced programming principles - data types, variables, control flow theory, compilers, loops, animation, game design, booleans, discrete numerical analysis - to middle school students on MIT’s Scratch IDE.}
        \item{Administered the after-school tutoring program by leading and training a group of Assistants.}
        \end{cvitems}
        \vspace{1em}
    }

\end{cventries}
