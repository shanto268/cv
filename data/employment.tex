\cvsection{Employment}

\begin{cventries}

\cventry
  {Levenson-Falk Lab (LFL), University of Southern California}
  {Graduate Research Assistant}
  {Los Angeles, CA, USA}
  {Jan. 2022 – Present}
  {
    \begin{cvitems}
      \item{Lead developer of \textit{SQuADDS}, an open-source platform for quantum device design, simulation, and fabrication-readiness; accelerated design iteration from weeks to minutes and adopted across multiple labs.}
      \item{Developed and validated custom high-yield fabrication processes for nanobridge-SQUID resonators (15 nm features), increasing functional device yield from <2\% to >90\%.}
      \item{Designed and fabricated superconducting circuits including nanobridge-based resonators, offset-charge-sensitive transmons, and custom Josephson parametric amplifiers for probing quasiparticle dynamics and enhancing qubit readout.}
      \item{Engineered full-stack experimental systems, including cryogenic infrastructure (Oxford Instruments, BlueFors), filtering and microwave chains, CPW PCB design, mounting, and advanced packaging techniques.}
      \item{Built automated measurement infrastructure using QUA, Labber, and AlazarTech acquisition systems; integrated parametric sweeps and custom calibration routines across multiple platforms.}
      \item{Created a highly accurate Hidden Markov Model (HMM)-based inference pipeline to extract real-time quasiparticle occupation states from I/Q trajectory data, enabling detailed dynamic modeling of QP trapping and release.}
      \item{Leading experiments on Andreev bound state spectroscopy and nanoSQUID-based QP traps to investigate and mitigate quasiparticle-induced decoherence in superconducting qubits.}
      \item{Extended SQuADDS to support interpretable ML workflows using Kolmogorov-Arnold Networks (KANs) to learn mapping between device geometry and target Hamiltonians.}
    %   \item{Published multiple papers on quasiparticle dynamics across device platforms, design workflows, and dissipation engineering.}
      \item{Mentored undergraduate researchers, master's students, and junior PhD students on quantum circuit design, measurement automation, and fabrication processes.}
    \end{cvitems}
  }

\cventry
  {Advanced Particle Detector Laboratory (APDL), Texas Tech University}
  {Undergraduate Research Assistant}
  {Lubbock, TX, USA}
  {Nov. 2018 – Aug. 2021}
  {
    \begin{cvitems}
      \item{Led end-to-end design of optical system upgrades for muon telescopes; developed custom Winston cones that improved signal efficiency from 20\% to 78\%.}
      \item{Co-designed and assembled both SiPM- and PMT-based muon telescopes; built DAQ systems using Arduino, CAMAC crates, and custom PCBs with wireless synchronization.}
      \item{Wrote real-time data acquisition and analysis software, converting raw readout into muon flux maps; deployed automated pipelines on the university HPC, with cloud-style report generation.}
      \item{Built and validated full Geant4-based Monte Carlo simulation of the experimental system, including physics modeling of photon scattering and muon interactions.}
      \item{Developed a custom ML architecture that inferred photon time-of-propagation to reconstruct 3D tomograms from 2D detector arrays.}
      \item{Actively involved in multiple data-taking campaigns, including first experimental run; maintained 24/7 operations and emergency response support.}
      \item{Presented work at national conferences, winning multiple awards for technical talks and posters on detector design and reconstruction algorithms.}
      \item{Co-authored peer-reviewed publications on both the initial prototype and next-generation telescope design.}
    \item{Supervisors: Shuichi Kunori, PhD. \& Nural Akchurin, PhD.}
    \end{cvitems}
  }

\cventry
  {Institute for Software Integrated Systems (ISIS), Vanderbilt University}
  {Summer Research Intern}
  {Nashville, TN, USA}
  {Jun. – Aug. 2020}
  {
    \begin{cvitems}
      \item{Built a full-stack calibration pipeline for microscopic traffic models, addressing parameter identifiability and stochastic noise under multi-objective constraints.}
      \item{Parallelized simulation-optimization workflows using Ray; achieved 10x+ speedup in sweep-based calibration experiments.}
      \item{Designed tools to convert simulation output from the Intelligent Driver Model (IDM) into radar-style datasets for validation against real-world aggregate metrics.}
      \item{Contributed to the Flow RL framework, enabling closed-loop learning in calibrated traffic environments; supported end-to-end tuning and evaluation.}
      \item{Co-authored a peer-reviewed conference paper on microsimulation calibration using aggregate measurements.}
    \item{Supervisors: Daniel Work, PhD. \& George Gunter, PhD.} 
    \end{cvitems}
  }


\cventry
  {Texas Tech Multidisciplinary Research in Transportation (TechMRT)}
  {Undergraduate Research Assistant}
  {Lubbock, TX, USA}
  {Jan. 2019 – Jun. 2020}
  {
    \begin{cvitems}
      \item{Developed an \href{https://github.com/shanto268/comprehensive_simulation_traffic_analysis_software}{open-source simulator} for heterogeneous AV/HV traffic using an extended Nagel-Schreckenberg CA model; supported both rule-based and learning agents.}
      \item{Designed AV control strategies for shared-lane mobility and dynamic lane switching; revealed emergent behaviors like intelligent herding and platoon formation.}
      \item{Integrated reinforcement learning for AVs to adapt to local density gradients; demonstrated benefits in flow stability and throughput in multi-lane networks.}
      \item{Analyzed macro-scale flow metrics derived from microscopic simulation rules; co-authored a journal paper identifying system-wide effects of AV/HV composition.}
      \item{Supervisor: Jia Li, PhD}
    \end{cvitems}
  }



\end{cventries}
